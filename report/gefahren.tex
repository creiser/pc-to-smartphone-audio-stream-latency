Die größte Gefahr für die interne Validität stellt die Qualität der Drahtlosübertragung dar. Eine schlechte Verbindungsqualität kann eine höhere Latenz verursachen. Mithilfe der Anwendung WiFi Analyzer wurde über den gesamten Testzeitraum sichergestellt, dass keine anderen WiFi-Sender wesentlich aktiv sind. Jedoch konnte damit nicht sichergestellt werden, dass Sender, die eine andere Drahtlosübertragungstechnik wie 4g oder Bluetooth verwenden, zu einer Verschlechterung der Verbindungsqualität beitragen. Weiter ist die interne Validität des Experiments dadurch gefährdet, dass die Zeitwerte vom Spektrogramm von unterschiedlichen Personen anders abgelesen werden könnten. Um die Objektivität des Messvorgangs zu erhöhen, wurden alle Latenzwerte von drei Personen unabhängig abgelesen. Die maximale Abweichung zwischen den Messergebnissen unterschiedlicher Personen betrug lediglich eine Millisekunden. Durch die Wiederholung der Tests mit Smartphones unterschiedlicher Leistungsklassen wurde versucht, das Ergebnis auf eine größere Bandbreite von Smartphones verallgemeinerbar zu machen. Um die externe Validität zu erhöhen, ist es jedoch sinnvoll mit einer größeren Anzahl verschiedener Smartphones zu testen.