Als klarer Sieger des Tests geht die App SoundWire hervor, da diese die geringste Latenz in allen Leistungsklassen aufweisen konnte. Wie auch schon in Kapitel \ref{threats} aufgezeigt, spielt die Qualität der WiFi-Verbindung eine entscheidende Rolle auf die Verzögerungszeit bei der drahtlosen Tonübertragung. In einem realen Umfeld, bei dem die Apps als Ersatz für Kabellose Kopfhörer zum Einsatz kommen sollen, muss klar nach Andwendungszweck unterschieden werden. Wird die Beschriebene Vorgehensweise beispielsweise zur Übertragung von Musik genutzt, stellt die Latenz keine Probleme dar. Die besten Latenzwerte die in unseren Tests gemessen wurden bewegten sich im Bereich knapp unter 100ms. Für Anwendungen im Filmbereich ist diese Latenz noch etwas zu groß. Eine Verzögerung von Ton- zur Videospur in dieser Größenordnung kann immer noch als störend empfunden werden. Es bleibt abzuwarten, ob in Zukunft Apps entwickelt werden, die diese Übertragungstechnik mit einer geringeren Latenz bewerkstelligen können. Außerdem ist es möglich, dass durch bessere WLAN-Technik eine Übertragung mit geringeren Verzögerungszeiten durchgeführt werden kann.