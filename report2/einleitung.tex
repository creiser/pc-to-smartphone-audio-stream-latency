Drahtlose Kopfhörer haben in den letzten Jahren an Beliebtheit gewonnen. Viele NutzerInnen besitzen jedoch bereits hochqualitative kabelgebundene Kopfhörer, wollen von den Vorteilen der Drahtlosigkeit profitieren, aber gleichzeitig nicht in neue Kopfhörer investieren. Abhilfe können Android-Anwendungen verschaffen, die die Tonausgabe des PCs der NutzerIn über WiFi an ihr Smartphone weiterleiten. An das Smartphone werden lediglich herkömmliche Kopfhörer angeschlossen. Die NutzerIn gewinnt damit Bewegungsfreiheit in einem ähnlichen Maße, wie es bei der Verwendung von drahtlosen Kopfhörern der Fall ist. Für das Ansehen von Videos, ist es erforderlich, dass Ton- und Bildausgabe möglichst synchron sind. Insbesondere bei Videos mit sprechenden Menschen kann es als störend empfunden werden, wenn der Ton verzögert abgespielt wird. Möchte die NutzerIn eine Übertragungsanwendung zum Ansehen von Filmen verwenden, ist es deshalb von zentraler Bedeutung, dass der Ton mit möglichst geringer Latenz abgespielt wird. Aus diesem Grund vergleichen wir in dieser Studie die Audiolatenz mehrerer Android-Anwendungen mit dem Ziel, diejenige mit der geringsten Latenz zu bestimmen und so der NutzerIn die Wahl einer geeigneten App zu erleichtern.